% Don't touch this %%%%%%%%%%%%%%%%%%%%%%%%%%%%%%%%%%%%%%%%%%%
\documentclass[11pt]{article}
\usepackage{fullpage}
\usepackage[left=1in,top=1in,right=1in,bottom=1in,headheight=3ex,headsep=3ex]{geometry}
\usepackage{graphicx}
\usepackage{float}
\usepackage{quoting}

\setlength{\parindent}{0em}
\setlength{\parskip}{1em}

\newcommand{\blankline}{\quad\pagebreak[2]}
%%%%%%%%%%%%%%%%%%%%%%%%%%%%%%%%%%%%%%%%%%%%%%%%%%%%%%%%%%%%%%

% Modify Course title, instructor name, semester here %%%%%%%%

\title{CS-271 System Programming, Computer Architecture}
\author{Calvin Deutschbein}
\date{Willamette University, Fall 2025}

%%%%%%%%%%%%%%%%%%%%%%%%%%%%%%%%%%%%%%%%%%%%%%%%%%%%%%%%%%%%%%

% Don't touch this %%%%%%%%%%%%%%%%%%%%%%%%%%%%%%%%%%%%%%%%%%%
\usepackage[sc]{mathpazo}
\linespread{1.05} % Palatino needs more leading (space between lines)
\usepackage[T1]{fontenc}
\usepackage[mmddyyyy]{datetime}% http://ctan.org/pkg/datetime
\usepackage{advdate}% http://ctan.org/pkg/advdate
\newdateformat{syldate}{\twodigit{\THEMONTH}/\twodigit{\THEDAY}}
\newsavebox{\MONDAY}\savebox{\MONDAY}{Mon}% Mon
\newcommand{\week}[1]{%
%  \cleardate{mydate}% Clear date
% \newdate{mydate}{\the\day}{\the\month}{\the\year}% Store date
  \paragraph*{\kern-2ex\quad #1, \syldate{\today} - \AdvanceDate[4]\syldate{\today}:}% Set heading  \quad #1
%  \setbox1=\hbox{\shortdayofweekname{\getdateday{mydate}}{\getdatemonth{mydate}}{\getdateyear{mydate}}}%
  \ifdim\wd1=\wd\MONDAY
    \AdvanceDate[7]
  \else
    \AdvanceDate[7]
  \fi%
}
\usepackage{setspace}
\usepackage{multicol}
%\usepackage{indentfirst}
\usepackage{fancyhdr,lastpage}
\usepackage{url}
\pagestyle{fancy}
\usepackage{hyperref}
\usepackage{lastpage}
\usepackage{amsmath}
\usepackage{layout}

\lhead{}
\chead{}
%%%%%%%%%%%%%%%%%%%%%%%%%%%%%%%%%%%%%%%%%%%%%%%%%%%%%%%%%%%%%%

% Modify header here %%%%%%%%%%%%%%%%%%%%%%%%%%%%%%%%%%%%%%%%%
\rhead{\footnotesize Thinking Machines}

%%%%%%%%%%%%%%%%%%%%%%%%%%%%%%%%%%%%%%%%%%%%%%%%%%%%%%%%%%%%%%
% Don't touch this %%%%%%%%%%%%%%%%%%%%%%%%%%%%%%%%%%%%%%%%%%%
\lfoot{}
\cfoot{\small \thepage/\pageref*{LastPage}}
\rfoot{}

\usepackage{array, xcolor}
\usepackage{color,hyperref}
\definecolor{clemsonorange}{HTML}{EA6A20}
\hypersetup{colorlinks,breaklinks,linkcolor=clemsonorange,urlcolor=clemsonorange,anchorcolor=clemsonorange,citecolor=black}

\begin{document}

\maketitle

\blankline

\begin{tabular*}{.93\textwidth}{@{\extracolsep{\fill}}lr}

%%%%%%%%%%%%%%%%%%%%%%%%%%%%%%%%%%%%%%%%%%%%%%%%%%%%%%%%%%%%%%

% Modify information %%%%%%%%%%%%%%%%%%%%%%%%%%%%%%%%%%%%%%%%%
E-mail: \texttt{ckdeutschbein@willamette.edu} & Web: \href{https://cd-public.github.io/courses/soc}{\tt\bf cd-public.github.io/}  \\

 Office Hours: MWF 11:30-12:00 \& Discord TTh  &  Lecture: MWF 1:10-2:40 \\

 Office: Ford 307 & Classroom: Ford 301  \\
 & \\
\hline
\end{tabular*}

\vspace{5 mm}

% First Section %%%%%%%%%%%%%%%%%%%%%%%%%%%%%%%%%%%%%%%%%%%%

\section*{Course Description}

\subsection*{College Colloquium}

Systems programmers study the boundary between abstractions in language and implementation at engineering and physical levels. This course will prepare computer scientists to reason at and across this abstraction boundary to more fully embrace the power of computation. Students will learn the systems language of Rust, systems libraries such as POSIX, and about UNIX-based systems such as Linux.

\section*{About Me}
Calvin Deutschbein is a 
fifth-year professor of computer science and colloquia instructor.

% Second Section %%%%%%%%%%%%%%%%%%%%%%%%%%%%%%%%%%%%%%%%%%%

\section*{Required Materials}

Required materials for a given class will be available on the \href{https://cd-public.github.io/scicom}{course webpage}. All course materials will be made available at no cost to the student.

\subsection*{Supplementary Text}

\href{https://doc.rust-lang.org/book/}{The Rust Programming Language} will be provided to students in electronic copy.


% Third Section %%%%%%%%%%%%%%%%%%%%%%%%%%%%%%%%%%%%%%%%%%%


\section*{Accessability}

I will make every effort to ensure all coursework and materials are accessible to all students, including working with on-campus specialists. However, there is always room for improvement. I always appreciate hearing from students about how I can make the course more accessible, so please reach out if there is something I can be doing better!

% Fourth Section %%%%%%%%%%%%%%%%%%%%%%%%%%%%%%%%%%%%%%%%%%%

\section*{Course Objectives}

\begin{itemize}
\item Write memory-safe code in compiled, non-garbage collected language.
\item Implement and use linear and hierarchical data structures and the basis of performant associative data structures.
\item Learn modern development practices in a state-of-the-art coding framework.
\item Utilize the command line and command line tools to create executables, including command line tools of your own.
\item Learn fundamental algorithms and use cases of major cryptographic technologies.
\item Survey number, set, graph, and automata theory in an applied setting.
\item See examples of cybersecurity principles including the CIA triad and threat models.
\end{itemize}

% Fifth Section %%%%%%%%%%%%%%%%%%%%%%%%%%%%%%%%%%%%%%%%%%%

\section*{Course Structure}

The course will be composed of lecture, labs, and hoemwork on systems c computing.

\subsection*{Class Structure}

Classes are scheduled for Monday and Wednesday at 13:10 PM. The lecture schedule is on the \href{https://cd-c89.github.io/rs}{course webpage}.

\subsubsection*{Midterm}

There will be an in-class code-based midterm, which will be ``open-everything" (except other students.

\subsubsection*{Final}

The final is the individual or partner development of a version control system command line tool in Rust.

\subsection*{Feedback and Grading}

\subsubsection*{Grading Scale}

We will vote in the first week of class whether to use ungrading or specification grading. Both candidates are described on the \href{https://cd-c89.github.io/rs}{course webpage}

\section*{Course Policies}

\subsection*{Collaboration Policy}

All collaboration is by mutual enthusiastic consent. Any collaboration is permitted on homeworks and labs, none on the midterm, and partners on the final. This will conducted via honor system unless I receive a (anonymous) student complaint.

\subsection*{AI Policy}

Use of anything considered AI is allowed but not recommended as I've tried using it and thought it was horrible. There are variety of good tools that actually work I would recommend instead, beginning with the tools installed in the first week of class.

\subsection*{College Policies}

The following material is adapted from ``Information for Syllabus'' recommended language on syllabus prepartion provided to insturctors in the College of Arts \& Sciences.

\subsubsection*{Academic Integrity}

Students of Willamette University are members of a community that values excellence and integrity in every aspect of life. As such, we expect all community members to live up to the highest standards of personal, ethical, and moral conduct. Students are expected not to engage in any type of academic or intellectually dishonest practice and encouraged to display honesty, trust, fairness, respect, and responsibility in all they do. Plagiarism and cheating are especially offensive to the integrity of courses in which they occur and against the College community as a whole. These acts involve intellectual dishonesty, deception, and fraud, which inhibit the honest exchange of ideas. Plagiarism and cheating may be grounds for failure in the course and/or dismissal from the College. \url{http://willamette.edu/cla/catalog/policies/plagiarism-cheating.php}

\subsubsection*{Commitment to Positive Sexual Ethics}

Willamette is a community committed to fostering safe, productive learning environments, and we value ethical sexual behaviors and standards. Title IX and our school policy prohibit discrimination on the basis of sex, which regards sexual misconduct — including discrimination, harassment, domestic and dating violence, sexual assault, and stalking. We understand that sexual violence can undermine students’ academic success, and we encourage affected students to talk to someone about their experiences and get the support they need. 

\begin{quoting}\textbf{Please be aware that as a mandatory reporter I am required to report any instances you disclose to Willamette's Title IX Coordinator.}\end{quoting}

If you would rather share information with a confidential employee who does not have this responsibility, please contact our confidential advocate at confidential-advocate@willamette.edu. Confidential support also can be found with SARAs and at the GRAC (503-851-4245); and at WUTalk - a 24-hour telephone crisis counseling support line (503-375-5353). If you are in immediate danger, you may reach campus safety at 503-370-6911.

\subsubsection*{DACA/Undocumented Student Advocate}

Willamette is committed to supporting our DACA/Undocumented students in a variety of ways. This year, Tori Ruiz is the contact person for all DACA/undocumented students can provide those students with a number of external and internal resources that are available. Her contact information is email:~\href{mailto:truiz@willamette.edu}{truiz@willamette.edu}, Office: 3rd Floor UC, Phone: 503-370-6447.

\subsubsection*{Diversity and Disability Statement}

Willamette University values diversity and inclusion; we are committed to a climate of mutual respect and full participation. My goal is to create a learning environment that is usable, equitable, inclusive and welcoming. If there are aspects of the instruction or design of this course that result in barriers to your inclusion or accurate assessment or achievement, please notify me as soon as possible. Students with disabilities are also encouraged to contact the Accessible Education Services office in Smullin 155 at 503-370-6737 or Accessible-info@willamette.edu to discuss a range of options to removing barriers in the course, including accommodations.

\subsubsection*{Religious Practice}

Willamette University recognizes the value of religious practice and strives to accommodate students’ commitment to their religious traditions whenever possible. Please let me know within the first two weeks of the semester if a conflict between holy days or other religious practice and full participation in the course is anticipated. I will do my best to work with you to determine a reasonable accommodation.

\textit{As an instructor, I will exercise my discretion to offer accomodations for conflicts after the first two weeks of the semester. You may always reach out to me, including retroactively, though the quality of the accomodation I am able to offer may improve given advanced warning!}

\subsubsection*{SOAR Center Offerings: Food, Clothing, and School Materials}

The Students Organizing for Access to Resources (SOAR) Center strives to create equitable access to food, professional clothing, commencement regalia, and scholarly resources for WU and Willamette Academy students. The SOAR Center is located on the Putnam University Center's third floor (in the former Women's Resource Center and across from the Harrison Conference Room). The space houses the Bearcat Pantry, Clothing Share, and First-Generation Book Drive and is maintained by committed students and staff and faculty advisers.

\subsubsection*{Trans Inclusion and Gender Justice}

I am always appreciative of the opportunity to address you by your affirming name or pronoun. Please advise me of the most affirming way to address you at any time so that I may do so.

If I ever misgender you in any way, I would greatly appreciate that you let me know, in whatever manner makes you comfortable, so that I can correct that error and endeavour to repair any harm. 

\subsubsection*{Mental Health}
As a student, you may experience a range of challenges that can interfere with learning, such as strained
relationships, increased anxiety, substance use, feeling down, difficulty concentrating and/or lack of
motivation. These mental health concerns or stressful events may diminish your academic performance and/or
reduce your ability to participate in daily activities. Willamette services are available and treatment does work.
If you think you need help, please contact Bishop Health as soon as possible at
\url{http://willamette.edu/offices/counseling/}. Crisis counseling is available 24/7 at WUTalk: 503-375-5353 and
Campus Safety is available at 503-370-6911. Emergency resources are also available from the Psychiatric
Crisis Center at 503-585-4949 and the National Suicide Prevention Lifeline at 1-800-273-8255.


\end{document}

